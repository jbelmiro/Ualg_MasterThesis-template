% This is the Reed College LaTeX thesis template. Most of the work
% for the document class was done by Sam Noble (SN), as well as this
% template. Later comments etc. by Ben Salzberg (BTS). Additional
% restructuring and APA support by Jess Youngberg (JY).
% Your comments and suggestions are more than welcome; please email
% them to cus@reed.edu
%
% See http://web.reed.edu/cis/help/latex.html for help. There are a
% great bunch of help pages there, with notes on
% getting started, bibtex, etc. Go there and read it if you're not
% already familiar with LaTeX.
%
% Any line that starts with a percent symbol is a comment.
% They won't show up in the document, and are useful for notes
% to yourself and explaining commands.
% Commenting also removes a line from the document;
% very handy for troubleshooting problems. -BTS

% As far as I know, this follows the requirements laid out in
% the 2002-2003 Senior Handbook. Ask a librarian to check the
% document before binding. -SN

%%
%% Preamble
%%
% \documentclass{<something>} must begin each LaTeX document
\documentclass[12pt,twoside]{reedthesis}
% Packages are extensions to the basic LaTeX functions. Whatever you
% want to typeset, there is probably a package out there for it.
% Chemistry (chemtex), screenplays, you name it.
% Check out CTAN to see: http://www.ctan.org/
%%
\usepackage{graphicx,latexsym}
\usepackage{amsmath}
\usepackage{amssymb,amsthm}
\usepackage{longtable,booktabs,setspace}
\usepackage{chemarr} %% Useful for one reaction arrow, useless if you're not a chem major
\usepackage[hyphens]{url}
% Added by CII
\usepackage{hyperref}
\usepackage{lmodern}
\usepackage{float}
\floatplacement{figure}{h}
% End of CII addition
\usepackage{rotating}

% Next line commented out by CII
%%% \usepackage{natbib}
% Comment out the natbib line above and uncomment the following two lines to use the new
% biblatex-chicago style, for Chicago A. Also make some changes at the end where the
% bibliography is included.
%\usepackage{biblatex-chicago}
%\bibliography{thesis}


% Added by CII (Thanks, Hadley!)
% Use ref for internal links
\renewcommand{\hyperref}[2][???]{\autoref{#1}}
\def\chapterautorefname{Chapter}
\def\sectionautorefname{Section}
\def\subsectionautorefname{Subsection}
% End of CII addition

% Added by CII
\usepackage{caption}
\captionsetup{width=5in}
% End of CII addition

% \usepackage{times} % other fonts are available like times, bookman, charter, palatino

% Syntax highlighting #22

% To pass between YAML and LaTeX the dollar signs are added by CII
\title{Thesis title}
\author{Name Surname}
% The month and year that you submit your FINAL draft TO THE LIBRARY (May or December)
\date{Month Year}
\division{Faculdade de Ciências Humanas e Sociais}
\advisor{Name Surname}
\institution{Universidade do Algarve}
\degree{Insert degree}
\year{2020}
\titulo{Titulo da tese}
%If you have two advisors for some reason, you can use the following
% Uncommented out by CII
% End of CII addition

%%% Remember to use the correct department!
\department{Departamento}
% if you're writing a thesis in an interdisciplinary major,
% uncomment the line below and change the text as appropriate.
% check the Senior Handbook if unsure.
%\thedivisionof{The Established Interdisciplinary Committee for}
% if you want the approval page to say "Approved for the Committee",
% uncomment the next line
%\approvedforthe{Committee}

% Added by CII
%%% Copied from knitr
%% maxwidth is the original width if it's less than linewidth
%% otherwise use linewidth (to make sure the graphics do not exceed the margin)
\makeatletter
\def\maxwidth{ %
  \ifdim\Gin@nat@width>\linewidth
    \linewidth
  \else
    \Gin@nat@width
  \fi
}
\makeatother

\renewcommand{\contentsname}{Table of Contents}
% End of CII addition

\setlength{\parskip}{0pt}

% Added by CII

\providecommand{\tightlist}{%
  \setlength{\itemsep}{0pt}\setlength{\parskip}{0pt}}

\Acknowledgements{

}

\Dedication{

}

\Preface{

}

\Abstract{
First paragraph, 300 words.

\par

Second paragraph.

~

\textbf{Keywords:} Example1; Example2
}

\Resumo{
Resumo em português com 1000 palavras, quando a tese é em Inglês.

\par

Segundo parágrafo do resumo.

\textbf{Palavras-chave:} Exemplo1; Exemplo2
}

	\usepackage{caption}
\usepackage{flafter}
\captionsetup[figure]{font=footnotesize}
\captionsetup[table]{font=footnotesize}
\usepackage{indentfirst}
\setlength{\parindent}{20pt}
% End of CII addition
%%
%% End Preamble
%%
%
\begin{document}

% Everything below added by CII

  \maketitle


\frontmatter % this stuff will be roman-numbered
%\pagestyle{empty} % this removes page numbers from the frontmatter


  \begin{abstract}
    First paragraph, 300 words.
    
    \par
    
    Second paragraph.
    
    ~
    
    \textbf{Keywords:} Example1; Example2
  \end{abstract}
  \begin{resumo}
    Resumo em português com 1000 palavras, quando a tese é em Inglês.
    
    \par
    
    Segundo parágrafo do resumo.
    
    \textbf{Palavras-chave:} Exemplo1; Exemplo2
  \end{resumo}
  \hypersetup{linkcolor=black}
  \setcounter{tocdepth}{2}
  \tableofcontents

  \listoftables

  \listoffigures


\mainmatter % here the regular arabic numbering starts
\pagestyle{fancyplain} % turns page numbering back on

\hypertarget{introduction}{%
\chapter{Introduction}\label{introduction}}

An introduction (Cascalheira and Bicho, 2013).

\hypertarget{methodology}{%
\chapter{Methodology}\label{methodology}}

A description of the methodology.

\hypertarget{results}{%
\chapter{Results}\label{results}}

Result description.

\hypertarget{discussion}{%
\chapter{Discussion}\label{discussion}}

Discussion of the results.

\hypertarget{conclusion}{%
\chapter{Conclusion}\label{conclusion}}

Brief overview of the thesis' main conclusions.

\hypertarget{references}{%
\chapter*{References}\label{references}}
\addcontentsline{toc}{chapter}{References}

\markboth{References}{References}

\noindent
\singlespacing

\setlength{\parindent}{-0.20in}
\setlength{\leftskip}{0.20in}
\setlength{\parskip}{8pt}

\hypertarget{refs}{}
\leavevmode\hypertarget{ref-cascalheiraandbicho2013}{}%
Cascalheira, J., Bicho, N., 2013. Hunter--gatherer ecodynamics and the impact of the heinrich event 2 in central and southern portugal. Quaternary International. 318, 117--127.

\appendix

\hypertarget{figures-and-tables}{%
\chapter{Figures and tables}\label{figures-and-tables}}

\hypertarget{colophon}{%
\chapter{Colophon}\label{colophon}}

This dissertation was generated on 2020-09-12 16:43:49 using the following computational environment and dependencies:
\begin{verbatim}
- Session info ---------------------------------------------------------------
 setting  value                       
 version  R version 3.6.2 (2019-12-12)
 os       Windows 10 x64              
 system   x86_64, mingw32             
 ui       RTerm                       
 language (EN)                        
 collate  English_United States.1252  
 ctype    English_United States.1252  
 tz       Europe/London               
 date     2020-09-12                  

- Packages -------------------------------------------------------------------
 package     * version date       lib source                            
 assertthat    0.2.1   2019-03-21 [1] CRAN (R 3.6.1)                    
 backports     1.1.5   2019-10-02 [1] CRAN (R 3.6.1)                    
 bookdown    * 0.16.5  2019-12-27 [1] Github (rstudio/bookdown@70f9c07) 
 callr         3.4.3   2020-03-28 [1] CRAN (R 3.6.3)                    
 cli           2.0.2   2020-02-28 [1] CRAN (R 3.6.3)                    
 colorspace    1.4-1   2019-03-18 [1] CRAN (R 3.6.1)                    
 crayon        1.3.4   2017-09-16 [1] CRAN (R 3.6.1)                    
 desc          1.2.0   2018-05-01 [1] CRAN (R 3.6.2)                    
 devtools    * 2.3.0   2020-04-10 [1] CRAN (R 3.6.2)                    
 digest        0.6.23  2019-11-23 [1] CRAN (R 3.6.2)                    
 dplyr       * 0.8.3   2019-07-04 [1] CRAN (R 3.6.1)                    
 ellipsis      0.3.0   2019-09-20 [1] CRAN (R 3.6.1)                    
 evaluate      0.14    2019-05-28 [1] CRAN (R 3.6.1)                    
 fansi         0.4.0   2018-10-05 [1] CRAN (R 3.6.1)                    
 fs            1.3.1   2019-05-06 [1] CRAN (R 3.6.2)                    
 ggplot2     * 3.2.1   2019-08-10 [1] CRAN (R 3.6.1)                    
 glue          1.3.1   2019-03-12 [1] CRAN (R 3.6.1)                    
 gtable        0.3.0   2019-03-25 [1] CRAN (R 3.6.1)                    
 htmltools     0.4.0   2019-10-04 [1] CRAN (R 3.6.1)                    
 knitr       * 1.26    2019-11-12 [1] CRAN (R 3.6.2)                    
 lazyeval      0.2.2   2019-03-15 [1] CRAN (R 3.6.1)                    
 lifecycle     0.2.0   2020-03-06 [1] CRAN (R 3.6.3)                    
 magrittr      1.5     2014-11-22 [1] CRAN (R 3.6.1)                    
 memoise       1.1.0   2017-04-21 [1] CRAN (R 3.6.2)                    
 munsell       0.5.0   2018-06-12 [1] CRAN (R 3.6.1)                    
 pillar        1.4.3   2019-12-20 [1] CRAN (R 3.6.2)                    
 pkgbuild      1.0.6   2019-10-09 [1] CRAN (R 3.6.2)                    
 pkgconfig     2.0.3   2019-09-22 [1] CRAN (R 3.6.1)                    
 pkgload       1.0.2   2018-10-29 [1] CRAN (R 3.6.2)                    
 prettyunits   1.0.2   2015-07-13 [1] CRAN (R 3.6.1)                    
 processx      3.4.1   2019-07-18 [1] CRAN (R 3.6.1)                    
 ps            1.3.0   2018-12-21 [1] CRAN (R 3.6.1)                    
 purrr         0.3.3   2019-10-18 [1] CRAN (R 3.6.1)                    
 R6            2.4.1   2019-11-12 [1] CRAN (R 3.6.2)                    
 Rcpp          1.0.3   2019-11-08 [1] CRAN (R 3.6.2)                    
 remotes       2.1.1   2020-02-15 [1] CRAN (R 3.6.3)                    
 rlang         0.4.5   2020-03-01 [1] CRAN (R 3.6.3)                    
 rmarkdown     2.1.2   2020-04-23 [1] Github (rstudio/rmarkdown@8aeaa6e)
 rprojroot     1.3-2   2018-01-03 [1] CRAN (R 3.6.2)                    
 rstudioapi    0.11    2020-02-07 [1] CRAN (R 3.6.3)                    
 scales        1.1.0   2019-11-18 [1] CRAN (R 3.6.2)                    
 sessioninfo   1.1.1   2018-11-05 [1] CRAN (R 3.6.2)                    
 stringi       1.4.3   2019-03-12 [1] CRAN (R 3.6.0)                    
 stringr       1.4.0   2019-02-10 [1] CRAN (R 3.6.1)                    
 testthat      2.3.2   2020-03-02 [1] CRAN (R 3.6.3)                    
 thesisdown  * 0.0.2   2019-12-27 [1] Github (ismayc/thesisdown@0024950)
 tibble        2.1.3   2019-06-06 [1] CRAN (R 3.6.1)                    
 tidyselect    0.2.5   2018-10-11 [1] CRAN (R 3.6.1)                    
 usethis     * 1.6.0   2020-04-09 [1] CRAN (R 3.6.3)                    
 withr         2.1.2   2018-03-15 [1] CRAN (R 3.6.1)                    
 xfun          0.11    2019-11-12 [1] CRAN (R 3.6.2)                    
 yaml          2.2.0   2018-07-25 [1] CRAN (R 3.6.0)                    

[1] C:/Users/Valentine/Documents/R/win-library/3.6
[2] C:/Program Files/R/R-3.6.2/library
\end{verbatim}

% Index?

\end{document}
